\documentclass[
	%parspace,	% Térköz bekezdések közé 					/ Add vertical space between paragraphs
	%noindent,	% Bekezdésének első sora ne legyen behúzva 	/ No indentation of first lines in each paragraph
	%nohyp,		% Szavak sorvégi elválasztásának tiltása 	/ No hypenation of words
	%twoside,	% Kétoldalas nyomtatás 						/ Double sided format
	%final,		% Teendők elrejtése 						/ Set final to hide todos
]{elteikthesis}[2019/09/28]

% Dolgozat metaadatai / Document's metadata
\title{Háromdimenziós közlekedés szimuláció} 	% cím / title
\date{2019} 									% védés éve / year of defense

% Szerző metaadatai / Author's metadata
\author{Sándor Balázs}
\degree{Programtervező Informatikus BSc}

% Témavezető(k) metaadatai
% Superivsor(s)' metadata
\supervisor{Baráth Dániel} % belső témavezető neve						/ internal supervisor's name
\affiliation{Doktorjelölt} % belső témavezető beosztása					/ internal supervisor's affiliation
%\extsupervisor{Külső Kornél} % külső témavezető neve 					/ external supervisor's name
%\extaffiliation{informatikai igazgató} % külső témavezető beosztása	/ external supervisor's affiliation

% Egyetem metaadatai
% University's metadata
\university{Eötvös Loránd Tudományegyetem}			% egyetem nev	/ university's name
\faculty{Informatikai Kar} 							% kar neve		/ faculty's name
\department{Algoritmusok és Alkalmazásaik tanszék}	% tanszék neve	/ department's name
\city{Budapest} 									% város 		/ city
\logo{elte_cimer_szines} 							% logo

% Irodalomjegyzék hozzáadása / Add bibliography file
\addbibresource{thesis.bib}

% A dolgozat / The document
\begin{document}
	
% Nyelv kiválasztása / Set document language
\documentlang{magyar}
%\documentlang{english}

% Teendők listája (final dokumentumban nincs) / List of todos (not in the final document)
\listoftodos[\todolabel]

% Dokumentum beállítások / Some document settings
% Lábjegyzet folytonos számozása fejezetek között	/ Contiunous counting of footnotes among chapters
\counterwithout{footnote}{chapter}

% Tartalomjegyzék oldalszámozásának rejtése 		/ Hide page numbering of ToC
\newcounter{conpageno}
\let\oldtableofcontents\tableofcontents
\renewcommand{\tableofcontents}{
	\pagenumbering{gobble}
	\oldtableofcontents
	\cleardoublepage
	\setcounter{conpageno}{\value{page}}
	\pagenumbering{arabic}
	\setcounter{page}{\value{conpageno}}
}

% Címlap (kötelező) / Title page (mandatory)
\maketitle
\topicdeclaration

% Tartalomjegyzék (kötelező) / Table of contents (mandatory)
\tableofcontents
\cleardoublepage

% Tartalom / Main content
\chapter{Bevezetés} % Introduction
\label{ch:intro}

Egy szimulációs eszköz elkészítése, mellyel útszakaszok forgalomáteresztő képessége vizsgálható.
A program beolvassa az útszakaszt és a forgalom paramétereit leíró fájlt, majd ezt háromdimenzióban megjeleníti.
A szimuláció során a program a felhasználó által definiált paramétereknek megfelelő forgalmat enged át a modellezett útszakaszon.

A járművek megközelítőleg a közúti közlekedés szabályainak megfelelően haladnak a belépési pontjuktól az előre definiált célállomásuk felé a saját útkereső algoritmusuk által kijelölt útvonal alapján.

Ha minden jármű célba ért, a szimuláció véget ér. Ekkor a program a megjeleníti a szimuláció során gyűjtött statisztikai adatokat. Így megállapíthatóvá válik, hogy a futtatott szimulációs és a megvalósított forgalmi szabályok milyen módon befolyásolják a forgalmat és a forgalmi fennakadásokat.

A program C++ nyelven kerül megvalósításra. A háromdimenziós megjelenítés OpenGL-el történik, aminek implementálásához az  SDL-t (Simple DirectMedia Layer) hívom segítségül.
A program fejlesztése és futtatása Windows platformon történik, de a fejlesztés során szemelőtt tartom, hogy a lehető leginkább platformfüggetlen megoldásokat alkalmazzak.

\cleardoublepage

\chapter{Felhasználói dokumentáció} % User guide
\label{ch:user}
\cleardoublepage

\chapter{Fejlesztői dokumentáció} % Developer guide
\label{ch:impl}
\cleardoublepage

\chapter{Összegzés} % Conclusion
\label{ch:sum}

Lorem ipsum dolor sit amet, consectetur adipiscing elit. In eu egestas mauris. Quisque nisl elit, varius in erat eu, dictum commodo lorem. Sed commodo libero et sem laoreet consectetur. Fusce ligula arcu, vestibulum et sodales vel, venenatis at velit. Aliquam erat volutpat. Proin condimentum accumsan velit id hendrerit. Cras egestas arcu quis felis placerat, ut sodales velit malesuada. Maecenas et turpis eu turpis placerat euismod. Maecenas a urna viverra, scelerisque nibh ut, malesuada ex.

Aliquam suscipit dignissim tempor. Praesent tortor libero, feugiat et tellus porttitor, malesuada eleifend felis. Orci varius natoque penatibus et magnis dis parturient montes, nascetur ridiculus mus. Nullam eleifend imperdiet lorem, sit amet imperdiet metus pellentesque vitae. Donec nec ligula urna. Aliquam bibendum tempor diam, sed lacinia eros dapibus id. Donec sed vehicula turpis. Aliquam hendrerit sed nulla vitae convallis. Etiam libero quam, pharetra ac est nec, sodales placerat augue. Praesent eu consequat purus.

\cleardoublepage

% Függelékek (opcionális)	- hosszabb részletező táblázatok, sok és/vagy nagy kép esetén hasznos
% Appendices (optional)		- useful for detailed information in long tables, many and/or large figures, etc.
\appendix
\chapter{Szimulációs eredmények} % Simulation results
\label{appx:simulation}

Lorem ipsum dolor sit amet, consectetur adipiscing elit. Pellentesque facilisis in nibh auctor molestie. Donec porta tortor mauris. Cras in lacus in purus ultricies blandit. Proin dolor erat, pulvinar posuere orci ac, eleifend ultrices libero. Donec elementum et elit a ullamcorper. Nunc tincidunt, lorem et consectetur tincidunt, ante sapien scelerisque neque, eu bibendum felis augue non est. Maecenas nibh arcu, ultrices et libero id, egestas tempus mauris. Etiam iaculis dui nec augue venenatis, fermentum posuere justo congue. Nullam sit amet porttitor sem, at porttitor augue. Proin bibendum justo at ornare efficitur. Donec tempor turpis ligula, vitae viverra felis finibus eu. Curabitur sed libero ac urna condimentum gravida. Donec tincidunt neque sit amet neque luctus auctor vel eget tortor. Integer dignissim, urna ut lobortis volutpat, justo nunc convallis diam, sit amet vulputate erat eros eu velit. Mauris porttitor dictum ante, commodo facilisis ex suscipit sed.

Sed egestas dapibus nisl, vitae fringilla justo. Donec eget condimentum lectus, molestie mattis nunc. Nulla ac faucibus dui. Nullam a congue erat. Ut accumsan sed sapien quis porttitor. Ut pellentesque, est ac posuere pulvinar, tortor mauris fermentum nulla, sit amet fringilla sapien sapien quis velit. Integer accumsan placerat lorem, eu aliquam urna consectetur eget. In ligula orci, dignissim sed consequat ac, porta at metus. Phasellus ipsum tellus, molestie ut lacus tempus, rutrum convallis elit. Suspendisse arcu orci, luctus vitae ultricies quis, bibendum sed elit. Vivamus at sem maximus leo placerat gravida semper vel mi. Etiam hendrerit sed massa ut lacinia. Morbi varius libero odio, sit amet auctor nunc interdum sit amet.

Aenean non mauris accumsan, rutrum nisi non, porttitor enim. Maecenas vel tortor ex. Proin vulputate tellus luctus egestas fermentum. In nec lobortis risus, sit amet tincidunt purus. Nam id turpis venenatis, vehicula nisl sed, ultricies nibh. Suspendisse in libero nec nisi tempor vestibulum. Integer eu dui congue enim venenatis lobortis. Donec sed elementum nunc. Nulla facilisi. Maecenas cursus id lorem et finibus. Sed fermentum molestie erat, nec tempor lorem facilisis cursus. In vel nulla id orci fringilla facilisis. Cras non bibendum odio, ac vestibulum ex. Donec turpis urna, tincidunt ut mi eu, finibus facilisis lorem. Praesent posuere nisl nec dui accumsan, sed interdum odio malesuada.
\cleardoublepage

% Irodalomjegyzék (kötelező) / Bibliography (mandatory)
\addcontentsline{toc}{chapter}{\biblabel}
\printbibliography[title=\biblabel]
\cleardoublepage

% Ábrajegyzék (opcionális) - 3-5 ábra fölött érdemes / List of figures (optional) - useful over 3-5 figures
\addcontentsline{toc}{chapter}{\lstfigurelabel}
\listoffigures
\cleardoublepage

% Táblázatjegyzék (opcionális) - 3-5 táblázat fölött érdemes / List of tables (optional) - useful over 3-5 tables
\addcontentsline{toc}{chapter}{\lsttablelabel}
\listoftables
\cleardoublepage

% Forráskódjegyzék (opcionális) - 3-5 kódpélda fölött érdemes / List of codes (optional) - useful over 3-5 code samples
\addcontentsline{toc}{chapter}{\lstcodelabel}
\lstlistoflistings
\cleardoublepage

% Jelölésjegyzék (opcionális)
% List of symbols (optional)
%\printnomenclature

\end{document}
