\chapter{Bevezetés}
\label{ch:Introduction}

Egy szimulációs eszköz elkészítése, mellyel útszakaszok forgalomáteresztő képessége vizsgálható.
A program beolvassa az útszakaszt és a forgalom paramétereit leíró fájlt, majd ezt háromdimenzióban megjeleníti.
A szimuláció során a program a felhasználó által definiált paramétereknek megfelelő forgalmat enged át a modellezett útszakaszon.

A járművek megközelítőleg a közúti közlekedés szabályainak megfelelően haladnak a belépési pontjuktól az előre definiált célállomásuk felé a saját útkereső algoritmusuk által kijelölt útvonal alapján.

Ha minden jármű célba ért, a szimuláció véget ér. Ekkor a program a megjeleníti a szimuláció során gyűjtött statisztikai adatokat. Így megállapíthatóvá válik, hogy a futtatott szimulációs és a megvalósított forgalmi szabályok milyen módon befolyásolják a forgalmat és a forgalmi fennakadásokat.

A program C++ nyelven kerül megvalósításra. A háromdimenziós megjelenítés OpenGL-el történik, aminek implementálásához az  SDL-t (Simple DirectMedia Layer) hívom segítségül.
A program fejlesztése és futtatása Windows platformon történik, de a fejlesztés során szemelőtt tartom, hogy a lehető leginkább platformfüggetlen megoldásokat alkalmazzak.
