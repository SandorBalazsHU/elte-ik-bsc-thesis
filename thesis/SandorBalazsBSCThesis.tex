\documentclass[
	%parspace,	% Térköz bekezdések közé 					/ Add vertical space between paragraphs
	%noindent,	% Bekezdésének első sora ne legyen behúzva 	/ No indentation of first lines in each paragraph
	%nohyp,		% Szavak sorvégi elválasztásának tiltása 	/ No hypenation of words
	%twoside,	% Kétoldalas nyomtatás 						/ Double sided format
	%final,		% Teendők elrejtése 						/ Set final to hide todos
]{SandorBalazsBSCThesis}[2019/10/04]

% Dolgozat metaadatai / Document's metadata
\title{Háromdimenziós közlekedés szimuláció} 	% cím / title
\date{2019} 									% védés éve / year of defense

% Szerző metaadatai / Author's metadata
\author{Sándor Balázs}
\degree{Programtervező Informatikus BSc}

% Témavezető(k) metaadatai
% Superivsor(s)' metadata
\supervisor{Baráth Dániel} % belső témavezető neve						/ internal supervisor's name
\affiliation{Doktorjelölt} % belső témavezető beosztása					/ internal supervisor's affiliation
%\extsupervisor{Külső Kornél} % külső témavezető neve 					/ external supervisor's name
%\extaffiliation{informatikai igazgató} % külső témavezető beosztása	/ external supervisor's affiliation

% Egyetem metaadatai
% University's metadata
\university{Eötvös Loránd Tudományegyetem}			% egyetem nev	/ university's name
\faculty{Informatikai Kar} 							% kar neve		/ faculty's name
\department{Algoritmusok és Alkalmazásaik tanszék}	% tanszék neve	/ department's name
\city{Budapest} 									% város 		/ city
\logo{elte_cimer_szines} 							% logo

% Irodalomjegyzék hozzáadása / Add bibliography file
\addbibresource{SandorBalazsBSCThesis.bib}

% A dolgozat / The document
\begin{document}
	
% Nyelv kiválasztása / Set document language
\documentlang{magyar}
%\documentlang{english}

% Teendők listája (final dokumentumban nincs) / List of todos (not in the final document)
\listoftodos[\todolabel]

% Dokumentum beállítások / Some document settings
\input{settings.tex}

% Címlap (kötelező) / Title page (mandatory)
\maketitle
\topicdeclaration

% Tartalomjegyzék (kötelező) / Table of contents (mandatory)
\tableofcontents
\cleardoublepage

% Tartalom / Main content
\chapter{Bevezetés}
\label{ch:Introduction}

Egy szimulációs eszköz elkészítése, mellyel útszakaszok forgalomáteresztő képessége vizsgálható.
A program beolvassa az útszakaszt és a forgalom paramétereit leíró fájlt, majd ezt háromdimenzióban megjeleníti.
A szimuláció során a program a felhasználó által definiált paramétereknek megfelelő forgalmat enged át a modellezett útszakaszon.

A járművek megközelítőleg a közúti közlekedés szabályainak megfelelően haladnak a belépési pontjuktól az előre definiált célállomásuk felé a saját útkereső algoritmusuk által kijelölt útvonal alapján.

Ha minden jármű célba ért, a szimuláció véget ér. Ekkor a program a megjeleníti a szimuláció során gyűjtött statisztikai adatokat. Így megállapíthatóvá válik, hogy a futtatott szimulációs és a megvalósított forgalmi szabályok milyen módon befolyásolják a forgalmat és a forgalmi fennakadásokat.

A program C++ nyelven kerül megvalósításra. A háromdimenziós megjelenítés OpenGL-el történik, aminek implementálásához az  SDL-t (Simple DirectMedia Layer) hívom segítségül.
A program fejlesztése és futtatása Windows platformon történik, de a fejlesztés során szemelőtt tartom, hogy a lehető leginkább platformfüggetlen megoldásokat alkalmazzak.

\cleardoublepage

\chapter{Felhasználói dokumentáció}
\label{ch:UserGuide}
\cleardoublepage

\chapter{Fejlesztői dokumentáció}
\label{ch:Implementation}
\cleardoublepage

\chapter{Összegzés}
\label{ch:Summary}
\cleardoublepage

% Irodalomjegyzék (kötelező) / Bibliography (mandatory)
\addcontentsline{toc}{chapter}{\biblabel}
\printbibliography[title=\biblabel]
\cleardoublepage

% Ábrajegyzék (opcionális) - 3-5 ábra fölött érdemes / List of figures (optional) - useful over 3-5 figures
\addcontentsline{toc}{chapter}{\lstfigurelabel}
\listoffigures
\cleardoublepage

% Táblázatjegyzék (opcionális) - 3-5 táblázat fölött érdemes / List of tables (optional) - useful over 3-5 tables
\addcontentsline{toc}{chapter}{\lsttablelabel}
\listoftables
\cleardoublepage

% Forráskódjegyzék (opcionális) - 3-5 kódpélda fölött érdemes / List of codes (optional) - useful over 3-5 code samples
\addcontentsline{toc}{chapter}{\lstcodelabel}
\lstlistoflistings
\cleardoublepage

\end{document}